\documentclass[11pt]{article} 
\usepackage{amsmath,amsthm,amstext,amssymb,amsfonts,amscd,graphicx,bbm,fullpage,url} 

%\voffset -0.5in
%\addtolength{\textheight}{1in} 

\textheight     8.75in
\textwidth      5.5in
\oddsidemargin  0.5in
%\topmargin     -0.25in
\nofiles
%\parskip 4pt


% new theorems
\newtheorem{lem}{Lemma} 
\newtheorem{lemma}{Lemma}
\newtheorem{thm}{Theorem} 
\newtheorem{ass}{Assumption} 
\newtheorem{cor}{Corollary} 
\newtheorem{clm}{Claim}
\newtheorem{prop}{Proposition} 
\newtheorem{con}{Conjecture} 
\newtheorem{wff}{Wff} 
\newtheorem{defn}{Definition} 
\newtheorem{axiom}{Axiom} 
\newcounter{rulenum} 
\newcounter{sent} 
\newcounter{tempcnt} 

% macros
\newcommand*{\add}{\mbox{\bf add}} 
\newcommand*{\open}{\mbox{\bf open}} 
\newcommand*{\close}{\mbox{\bf close}} 
\newcommand*{\ex}{\mbox{\rm E}} 
\newcommand*{\pr}{\mbox{\rm Pr}} 
\newcommand*{\range}{\mbox{\rm range}} 
\newcommand*{\rank}{\mbox{\rm rank}} 
\newcommand*{\sgn}{\mbox{\rm sign}} 
\newcommand*{\var}{\mbox{\rm Var}} 
\newcommand*{\diag}{\mbox{\rm diag}} 
\newcommand*{\epi}{\epsilon} 
\newcommand*{\QED}{\ \hfill\rule[-2pt]{6pt}{12pt} \medskip}
\newcommand*{\supp}{\mbox{\rm supp}} 

\newcommand*{\grad}{\nabla}
\newcommand*{\half}{\frac{1}{2}}
\newcommand*{\inv}{^{-1}}
\newcommand*{\0}{\mathbf{0}}
\newcommand*{\1}{\mathbf{1}}
\newcommand*{\E}{\ensuremath{\operatorname{E}}}
\newcommand*{\maximize}{\text{maximize}}
\newcommand*{\minimize}{\text{minimize}}
\newcommand*{\st}{\text{subject to}}
\newcommand*{\R}{\mathbbm{R}}
\newcommand*{\matlab}{{\sc Matlab}}

\newcommand{\abs}[1]{\left\vert #1 \right\vert}
\newcommand{\bigo}[1]{\mathcal{O} \left( #1 \right)}
\newcommand{\cov}[2]{\ensuremath{\operatorname{Cov}\left( #1, #2\right)}}
\newcommand{\Ex}[2][]{\ensuremath{\E_{#1} \left[ #2 \right]}}
\newcommand{\norm}[1]{\left\lVert\,#1\,\right\rVert}
\newcommand{\bmat}[1]{\begin{bmatrix}#1\end{bmatrix}}
\newcommand{\pmat}[1]{\begin{pmatrix}#1\end{pmatrix}}
\newcommand{\smallmat}[1]{\left (\begin{smallmatrix}#1\end{smallmatrix} \right)}
\newcommand{\vb}[1]{\mathbf{#1}}

\renewcommand{\P}{\ensuremath{\operatorname{P}}}
\renewcommand{\Pr}[2][]{\ensuremath{\P_{#1} \left \{ #2 \right \}}}


\date{}
\thispagestyle{empty}

\begin{document}


\subsection*{Program title}

Numerical Optimization Algorithms and Software for Systems Biology


\subsection*{Poster title}

Existence of Positive Equilibria for Mass Conserving and Mass-Action
Biochemical Reaction Networks with a Single-Terminal-Linkage Class


\subsection*{Authors and affiliations}

Santiago Akle,$^*$\footnote{Institute for Computational and Mathematical Engineering,
           Stanford University, Stanford, CA.
           Research supported in part by DOE Grant DE-SC0002009.
           \url{akle@stanford.edu}, \url{ntaheri@stanford.edu}, \url{onkar@stanford.edu}.}
Onkar Dalal,\footnotemark[1]
Ronan Fleming,\footnote{Science Institute \& Center for Systems
           Biology, University of Iceland, Reykjavik.
           \url{ronan.mt.fleming@gmail.com}.}
{\bf Michael Saunders},\footnotemark[3]
Nicole Taheri,\footnotemark[1]
{\bf Yinyu Ye}\footnote{Department of Management Science and Engineering,
           Stanford University, Stanford, CA.
           Research supported in part by NSF Grant GOALI 0800151 and DOE Grant DE-SC0002009.
           \url{saunders@stanford.edu}, \url{yinyu-ye@stanford.edu}.}
        
      
\subsection*{Abstract}

A steady state of a chemical reaction network is a set of chemical
concentrations that remain constant for the induced reaction rates.
In this work we assume the \emph{law of mass-action} governs the rate
of the reactions, i.e.\ the rate of a reaction is proportional to the
concentrations of the participating species. More specifically, if
$Y_{ij}$ is the stoichiometric coefficient for species $i$ in reaction
$j$, $k_{j}$ is the thermodynamically feasible rate constant for
reaction $j$, and $c_i$ is the concentration of species $i$, then the
rate of reaction $j$ is
\[v_j = k_j\prod_ic_i^{Y_{ij}}.\]

Assuming the reactions are \emph{mass conserving}, and that the
directed graph corresponding to the set of reactions forms a strongly
connected component, we show that, regardless of the rate constants,
there exists at least one steady state where all concentrations are
positive.

We also establish the parallel between steady states and a fixed point
of a mapping that arises from solving a strictly convex optimization
problem, which allows us to find such steady states in randomly
constructed networks.

\end{document}

