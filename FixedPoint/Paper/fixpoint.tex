\documentclass[smallextended]{svjour3}       % onecolumn (second format)

\smartqed  % flush right qed marks, e.g. at end of proof

\usepackage{graphicx,bbm,algorithmic,algorithm} 
\usepackage{amsmath,amstext,amssymb,amsfonts,amscd}
%\usepackage{amsthm}

% new theorems
% \newtheorem{lem}{Lemma} 
% \newtheorem{lemma}{Lemma}
% \newtheorem{thm}{Theorem} 
% \newtheorem{ass}{Assumption} 
% \newtheorem{cor}{Corollary} 
% \newtheorem{clm}{Claim}
% \newtheorem{prop}{Proposition} 
% \newtheorem{con}{Conjecture} 
% \newtheorem{wff}{Wff} 
 \spnewtheorem{defn}{Definition}{\bf}{\rm} 
% \newtheorem{axiom}{Axiom} 
\newcounter{rulenum} 
\newcounter{sent} 
\newcounter{tempcnt} 

% macros
\newcommand*{\add}{\mbox{\bf add}} 
\newcommand*{\open}{\mbox{\bf open}} 
\newcommand*{\close}{\mbox{\bf close}} 
\newcommand*{\ex}{\mbox{\rm E}} 
\newcommand*{\pr}{\mbox{\rm Pr}} 
\newcommand*{\range}{\mbox{\rm range}} 
\newcommand*{\rank}{\mbox{\rm rank}} 
\newcommand*{\sgn}{\mbox{\rm sign}} 
\newcommand*{\var}{\mbox{\rm Var}} 
\newcommand*{\diag}{\mbox{\rm diag}} 
\newcommand*{\epi}{\epsilon} 
\newcommand*{\QED}{\ \hfill\rule[-2pt]{6pt}{12pt} \medskip}
\newcommand*{\supp}{\mbox{\rm supp}} 

\newcommand*{\grad}{\nabla}
\newcommand*{\half}{\frac{1}{2}}
\newcommand*{\inv}{^{-1}}
\newcommand*{\0}{\mathbf{0}}
\newcommand*{\1}{\mathbf{1}}
\newcommand*{\E}{\ensuremath{\operatorname{E}}}
\newcommand*{\maximize}{\text{maximize}}
\newcommand*{\minimize}{\text{minimize}}
\newcommand*{\st}{\text{subject to}}
\newcommand*{\R}{\mathbbm{R}}
\newcommand*{\matlab}{{\sc Matlab}}

\newcommand{\abs}[1]{\left\vert #1 \right\vert}
\newcommand{\bigo}[1]{\mathcal{O} \left( #1 \right)}
\newcommand{\cov}[2]{\ensuremath{\operatorname{Cov}\left( #1, #2\right)}}
\newcommand{\Ex}[2][]{\ensuremath{\E_{#1} \left[ #2 \right]}}
\newcommand{\norm}[1]{\left\lVert\,#1\,\right\rVert}
\newcommand{\bmat}[1]{\begin{bmatrix}#1\end{bmatrix}}
\newcommand{\pmat}[1]{\begin{pmatrix}#1\end{pmatrix}}
\newcommand{\smallmat}[1]{\left (\begin{smallmatrix}#1\end{smallmatrix} \right)}

\renewcommand{\P}{\ensuremath{\operatorname{P}}}
\renewcommand{\Pr}[2][]{\ensuremath{\P_{#1} \left \{ #2 \right \}}}

\newenvironment{opt}{\begin{equation*} \begin{array}{lll}}{\end{array}\end{equation*}}

\title{Existence of positive steady states for mass-conserving and mass-action
	weakly reversible chemical reaction networks with a single linkage class%
   %\thanks{Submitted October 24, 2011 to Journal of Mathematical Biology.}
}

\author{Santiago Akle%
   % \thanks{Institute for Computational and Mathematical Engineering,
   %         Stanford University, Stanford, CA 94305.
   %         Research supported in part by the U.S. Department of Energy (Office
   %         of Advanced Scientific Computing Research and Office of Biological
   %         and Environmental Research) as part of the Scientific Discovery
   %         Through Advanced Computing program, grant DE-SC0002009.
   %         Email: {\tt akle@stanford.edu}, {\tt onkar@stanford.edu}, 
   %                {\tt ntaheri@stanford.edu}.}
   \and Onkar Dalal %\footnotemark[2]
   \and Ronan~Fleming%
   %\thanks{Center for Systems Biology, University of
   %Iceland, Sturlugata 8, Reykjavik 101, Iceland. Email: {\tt
   % ronan.mt.fleming@gmail.com}}
   \and Michael~Saunders% \footnotemark[4] 
   \and Nicole Taheri% \footnotemark[2]
   \and Yinyu Ye%
   % \thanks{Department of Management Science and Engineering,
   %         Stanford University, Stanford, CA 94305. Research supported in part by NSF grant GOALI 0800151
   %         and DOE grant DE-SC0002009.
   %         Email: {\tt saunders@stanford.edu}, {\tt yinyu-ye@stanford.edu}.}
}

\institute{S. Akle \at
           ICME, Stanford University, Stanford, CA 94305
        \\ \email{akle@stanford.edu}
           \and
           O. A. Dalal \at
           ICME, Stanford University, Stanford, CA 94305
        \\ \email{onkar@stanford.edu}
           \and
           R. M. T. Fleming \at
           Center for Systems Biology, University of
	 Iceland, Sturlugata 8, Reykjavik 101, Iceland
      \\ \email{ronan.mt.fleming@gmail.com}
         \and
           M. A. Saunders \at
           Dept of Management Science and Engineering, Stanford University, Stanford, CA 94305
        \\ \email{saunders@stanford.edu}
           \and
           N. A. Taheri \at
           ICME, Stanford University, Stanford, CA 94305
        \\ \email{ntaheri@stanford.edu}
         \and
           Y. Ye \at
           Dept of Management Science and Engineering, Stanford University, Stanford, CA 94305
        \\ \email{yinyu-ye@stanford.edu}
}

\titlerunning{Existence of positive steady states}        % if too long for running head
\authorrunning{S. Akle et al.}

\date{\today}


\begin{document}
\maketitle


\begin{abstract}
	We establish that mass-conserving 
    weakly reversible chemical reaction networks formed by a single linkage class
    admit positive steady states, regardless of network deficiency and choice
    of reaction rate constants. This result holds for closed systems (those
    without material exchange across the boundary) as well as for open systems 
    with material exchange at rates that satisfy a simple necessary and sufficient
    condition. Our proof uses a convex analysis formulation to define a mapping
    for which a positive fixed point is shown to exist. We then use the existence
    of the fixed point to establish the existence of a positive steady state.

    The proof inspires the definition of a practical algorithm to find these
    steady states. The same algorithm has been consistently successful
    at finding positive fixed points of weakly reversible networks with 
    \emph{multiple} linkage classes.
    %       We report the results of our algorithm on numerical experiments.
        We report numerical experiments.
   \keywords{systems biology networks \and dynamical systems \and fixed point}
   \subclass{92C42 \and 37N25 \and 65K10}
\end{abstract}

%\enlargethispage{1\baselineskip}

\section{Introduction}
Chemical reaction network theory (CRNT) studies the behavior of ensembles of chemical
reactions. One of the models considered by CRNT assumes that the rates of reaction
follow mass action kinetics \cite{GMAK}. The behavior of these systems depends 
on kinetic parameters.
However, measuring these parameters
experimentally is difficult and error-prone.  Thus, we seek properties of
chemical reaction networks that are independent of them.

In this work, we address the issue of existence of positive steady states,
i.e., positive concentrations of species that will stay constant under the
system's dynamics.  In particular, we tackle the case in which a directed graph
of chemical reactions forms a \emph{strongly connected component}, and the
reactions conserve mass. We prove the existence of positive equilibria for closed
networks (those without material exchange across their boundary), and extend our 
methods to open systems where species are exchanged across the boundary at certain rates.
How best to compute such steady states remains uncertain; however, we suggest a fixed
point algorithm and provide results of our numerical experiments for several cases. 

We draw from the work of Flemming et al. \cire{fleming-opt}
where the authors address case of ensembles of reversible reactions. They show that 
for such networks there exist chemical potentials which induce fluxes that 
are both thermodynamically feasible and mass conserving. And propose
an algorithm to calculate them. From that work we draw inspiration for using a 
linearly constrained convex optimization formulation which 
we extend to the case of weakly reversible networks. 

We wish to highlight the differences between our work and that of Deng et al.\ 
\cite{Deng}. Their work proves the existence of positive steady states for 
weakly reversible networks regardless of the number of connected components.
In this sense their proof is more general. However, we
also address the case when the rates of material exchange across the system's 
boundary are prescribed a priori and show that for one linkage class, weakly 
reversible chemical reaction networks, there exists a positive steady state if
the exchange rate satisfies a necessary and sufficient condition. Furthermore, we 
define a practical algorithm to find steady states of the system.

\subsection{Organization of the paper}

Rather than trying to review existing results in the area, we direct the 
interested reader to \cite{gunawardena,GMAK} and the work by Feinberg et al.\ 
\cite{deficiency0,deficiency1}. In section \ref{section:crnt-model} we establish
the notation and define the CRNT model that we address. In section
\ref{section:fp-model} we cover our proof, and in section \ref{sec:experiments}
we define the algorithm, evaluate its behavior, and make some final remarks.

%
%\subsection{Background}
%\label{background}
%
%\textit{Chemical Reaction Network Theory} (CRNT), a mathematical theory for
%this problem, has its roots in the seminal work by Fritz Horn, Roy Jackson, and
%Martin Feinberg in \cite{GMAK,uniqueEPandLyapunov,necc-suff-CB,deficiency0,deficiency1}.
%We build on the notation used in these works and summarized by
%Gunawardena\ in \cite{gunawardena}. The system consists of a collection
%of species reacting collectively in some combination to give another
%combination of species in a network of chemical reactions.  Let $\mathcal{S}$
%be a set of $m$ species and $\mathcal{C}$ be a set of $n$ complexes. The
%relation between species and complexes can be written as a non-negative matrix
%$Y\in\R^{m\times n}$, where column $y_j$ represents complex $j$, and $Y_{i j}$
%is the multiplicity of species $i$ in complex $j$.  For example, the
%multiplicity of the species NaCl in the complex (H$_2$O + 2NaCl) is 2.  

\section{A Model of Chemical Reaction Networks Driven by Mass-Action Kinetics} 
\label{section:crnt-model}
A reaction network is represented by a weighted directed graph
$G(V,E)$, where each node in $V$ represents a complex, each directed edge
$i\rightarrow j$ denotes a reaction using $i$ to generate $j$, and the
positive edge weight $k_{i\rightarrow j}$ is the reaction rate. The matrix $A
\in \R^{n \times n}$ is the weighted adjacency matrix of the graph, where
$A_{ij}=k_{i\rightarrow j}$.  Define $D := \diag(A\1)$, where $\1$ is the
vector of all ones, and $K := A^T-D$.  The elements of this new matrix will
be $(K)_{ij} = k_{j \rightarrow i}$ for $i \neq j$, and $(K)_{ii} = -\sum_j
k_{j \rightarrow i}$, so that $K^T \1 = \0$.  We assume that each complex
is used in a reaction. Hence $A$ will have no empty rows and $D$ will 
be a positive-definite diagonal matrix.

Let $c \in \R^m$ be the vector of concentrations of each species, and $b \in
\R^m$ the vector of species' exchange rates across the network boundary.
%We
%establish necessary and sufficient conditions on the external exchange rates
%$b$ in order to determine the existence of steady state concentrations, and to
%be able to compute it given a specific $b$.  

We define $\psi(c):\R^m_+\rightarrow \R^n$ to be a nonlinear function that
captures mass-action kinetics:
\[
\psi_j(c) = \prod_i\,c_i^{Y_{ij}},
\]
where $Y_{ij}$ is the stoichiometric coefficient of species $i$ in complex $j$.
The change in concentration over time can be described by the system of
ordinary differential equations 
\[
\dot{c} = YK\psi(c) - b.
\] 
\noindent Hence, steady state concentrations for a chemical reaction network
are any non-negative vector $c^\star\in\R^m$ such that $YK\psi(c^\star)=b$.
Equivalently, a pair $(c^\star,v^\star)$ with $c^\star\in\R^m$ and
$v^\star\in\R^n$ will be a steady state if it satisfies the conditions
\begin{align} 
  YKv^\star &=b \label{fb}\tag{FB} \\ \psi(c^\star) &= v^\star
  \label{mak}\tag{MA} 
\end{align}
or \emph{flux-balance} and \emph{mass-action}, respectively. Thus, finding 
steady state concentrations is equivalent to finding a vector $v^\star$ of potentials
that satisfies both $\eqref{fb}$ and \eqref{mak} for some vector of concentrations 
$c^\star\in\R^m$.

Observe that if $(c^\star,v^\star)$ is a positive steady state, then from the
definition of $\psi(c^\star)$,
\begin{align}
  Y^T\log(c^\star)= \log(\psi(c^\star))= \log(v^\star).
  \label{mak-alt}
	\tag{MA-log}
\end{align} 
We refer to this alternative condition as the logarithmic form of the
mass-action condition \eqref{mak}. 
%In this notation, systems with the property
%of mass conservation can be characterized by the following definition. 

\begin{defn}
	A chemical reaction network is $\emph{mass-conserving}$ if and only if there
	exists a positive vector $e\in\R^m$ such that 
	\begin{align}
	 e^TYK=0,
	  \label{consis}
	\end{align}
	where $e$ denotes the molecular weights of the species (or atomic weights if
	the species are elements).  
\end{defn}

%The connectedness of the networks is captured in the following definition of a
%terminal-linkage class.  
%\begin{defn} 
%	A \emph{terminal-linkage class} is defined as a set of complexes
%	$\mathcal{L}$ such that for any pair of complexes $(i,j) \in \mathcal{L}$
%	there exists a directed path in the graph $G$ that leads from $i$ to $j$.  
%\end{defn} 

%We further restrict our analysis to a class of weakly reversible networks.
\begin{defn} 
	A chemical reaction network is \emph{weakly reversible} if for any
    complex $i$ there exists a directed path in the graph $G$ that loops
    from $i$ to itself.
\end{defn} 

\begin{defn} 
    A \emph{linkage class} is a set of complexes $\mathcal{L}$ such that for any pair of complexes
    $(i,j) \in \mathcal{L}$ there exists a directed path in $G$ from $i$ to $j$. 
\end{defn}

From these definitions one can deduce that the complexes of a weakly reversible 
reaction network must partition into linkage classes.

%A reaction network that consists of exactly one terminal-linkage class is
%called a \emph{single terminal-linkage network}. Reversibility, at least in a
%weak sense, is a prerequisite for steady states with positive concentrations
%for all species, as suggested by simple examples like a single non-reversible
%reaction.
%
%Next we define a stoichiometric subspace and deficiency for a network.
%\begin{defn} 
%	A \emph{stoichiometric subspace} $S$ is the subspace defined by the span of
%	vectors $y_{j}-y_{i}$, where $y_{j}, y_{i}$ are the columns of $Y$
%	representing complexes $i$ and $j$, for each reaction pair $i \rightarrow j$
%	in the network. 
%\end{defn} 
%
%\begin{defn}
%	The \emph{deficiency} of a network is defined as $\delta = n - t - s$, where $t$ is
%	the number of terminal-linkage classes and $s$ is the dimension of the
%	stoichiometric subspace, also known as the stoichiometric compatibility class.
%\end{defn}
%
%In one of their early works, Horn and Jackson \cite{GMAK} analyzed mass-action
%kinetics for closed systems (with $b=0$) and defined a class of equilibrium
%points called \emph{complex-balanced equilibria}, and defined systems admitting
%such an equilibrium to be \emph{complex-balanced systems}. These closed systems
%are shown to satisfy the \emph{quasi-thermostatic} and
%\emph{quasi-thermodynamic} conditions regardless of the kinetic rate constants.
%Following this, Horn \cite{necc-suff-CB} also proved necessary and sufficient
%conditions for existence of a \emph{complex-balanced} equilibrium. In
%\cite{uniqueEPandLyapunov}, Feinberg and Horn used the existence of a Lyapunov
%function to show the uniqueness of the positive steady state in each
%stoichiometric compatibility class, which is equivalent to specifying all the
%conserved quantities of a system. Later Feinberg \cite{deficiency0,deficiency1}
%proved two theorems, now famously known as Deficiency 0-1
%theorems, that provide the analysis of positive steady states for a
%class of networks with deficiency $0$ or networks with deficiency $1$ but with
%each terminal-linkage class having deficiency less than $1$. For this
%restricted class of closed systems, the existence of a positive steady state is
%given by Perron-Frobenius theory for a positive eigenvector.  Other work in
%this area is from the perspective of dynamical systems and aimed toward proving
%two open conjectures: \emph{Global Attractor Conjecture} and \emph{Persistence
%Conjecture} \cite{Anderson-GAC}. Another approach using parametrized convex
%optimization to compute a non-equilibrium steady state is given in
%\cite{fleming-opt}.
%
%To the best of our knowledge, the vast majority of CRNT research studies closed
%\textit{complex-balanced systems}, which, by definition, admit a
%\textit{complex-balanced equilibrium}. In the notation above, a
%complex-balanced equilibrium exists when the vector of concentrations $c$ satisfies
%$A_k\psi(c)=0$, i.e., the vector $\psi(c)$ belongs to the null space of $A_k$.
%It can be shown that a network with linearly independent complexes will have
%deficiency $\delta = 0$. However, if some of the complexes are linearly
%dependent (as shown in the example in Sect.~\ref{sec:experiments}),
%there are systems that are
%not complex-balanced yet admit concentrations in equilibria where $A_k \psi(c)$
%is in the null space of $Y$ and $A_k\psi(c) \neq 0$.  Though the condition of
%\emph{complex-balance} is sufficient for \emph{thermodynamic} consistency,
%\cite{GMAK} shows that it is not necessary. Also, for open systems with
%material exchange across the boundary, \emph{complex-balance} is not defined.
%In order to handle open systems, these works hint at extending the system using
%a \emph{pseudo $0$-complex} and adding \emph{pseudo reactions}.  However, it is
%unclear how to choose the \emph{pseudo kinetic rates} such that the positive
%eigenvector solution of the extended system will achieve the given external
%exchange rates $b$. From the point of view of systems biology and bio-chemical
%engineering, analyzing the behavior of a cell under different exchange
%conditions $b$ is very important to control and engineer the cell, for example
%studying the desired effects in pharmacology, or producing specific metabolites
%in bioreactors.
%
%\vspace{0.2in}
%In this paper, we extend the previous work on two accounts: 1) we prove
%existence of positive equilibria in closed systems for some reaction networks
%that do not satisfy the necessary conditions of the Deficiency 0-1 theorems
%(are not necessarily \emph{complex-balanced}), and 2) we provide a necessary
%and sufficient condition on the external exchange rate $b$ for some open
%systems to admit a positive steady state. We use a fixed point of a convex
%optimization problem, with an objective function similar to the Helmholtz
%function defined in \cite{GMAK}. The fixed point of this mapping gives the
%required steady state. We prove the existence of a positive steady state for
%any weakly reversible chemical reaction network with a \emph{single
%terminal-linkage class}. We strongly believe that this can be extended to
%systems with \emph{multiple terminal-linkage classes}, as supported by our
%computational results for randomly generated networks. Sect.~\ref{example-network}
%gives a detailed analysis of a toy network to emphasize this claim.
%

\section{A fixed-point model} 
\label{section:fp-model}

Our main result establishes that for any set of positive reaction rates
$k\in\R^{|E|}$ and any $b$ in the range of $YK$, a weakly reversible network with 
a single linkage class will admit a positive solution pair $(c,v)$ that satisfies the laws
\eqref{fb} and \eqref{mak}.  
%We show this by defining a positive fixed point of
%a convex optimization problem, and establishing an equivalence between the
%positive fixed point and positive solution to the equations.

We construct a linearly constrained convex optimization
problem such that the logarithmic form of the mass-action equation
\eqref{mak-alt} is an optimality condition, so that any solution to this
optimization problem will also satisfy \eqref{mak-alt}.

We use the existence and uniqueness of the solution of the convex problem 
to define a mapping as follows: Let $b=YK\eta$ for some $\eta \in \R^n$ and observe
that for arbitrary $s \in \R^n$ we can write $b = YA^T(\eta +s) - YD(\eta +
D^{-1}A^Ts)$.  In particular, we choose $s$ positive and large enough so
that $\eta^+ := \eta+s$ and $\eta^- := \eta + D^{-1}A^Ts$ are both positive.
Also, from Definition \ref{consis}, $e^Tb = 0$ and thus
\begin{equation}
  e^TYD\eta^- = e^TYA^T\eta^+.
  \label{massbalanceb}
\end{equation}
Define $\mu:=(r,r_0) \in \R^{m+1}$ to be a vector parameter.  Observe that if
the parametric convex optimization problem
\begin{equation}
  \tag{P0}
	\begin{array}{lll}
  \underset{(v,v_0)\in\R^{n+1}}{\minimize} & v^T D(\log(v)-\1) + v_0(\log v_0 -1) \\
	\st &  YD v + YA^T\eta^+ v_0 = YA^Tr + YD\eta^-r_0 &:\ y \\
	    &\quad (v,v_0) \ge 0                     
	\end{array}
	\label{convex-fix}
\end{equation}
has a positive solution $(v^\star(\mu),v_0^\star(\mu))$, then the optimality
conditions
\begin{subequations}
    \label{opt-cond}
	\begin{align}
      YDv^\star(\mu)  + YA^T\eta^+ v^\star_0  &= YA^Tr + YD\eta^-r_0  \label{opt-primal-feas}\\
      DY^T y^\star(\mu) &= D\log(v^\star(\mu))                        \label{opt-dual-feas}\\
      (YA^T\eta^+)^Ty^\star(\mu)   &= \log(v^\star_0(\mu))            \label{opt-dual-feas-two}\\
    (v^\star(\mu),v_0^\star(\mu)) &\ge 0                              \label{opt-primal-bounds}
	\end{align}
\end{subequations}
are well defined. Since $D$ is nonsingular, \eqref{opt-dual-feas} is
equivalent to \eqref{mak-alt} with $c^\star(\mu) := e^{y^\star(\mu)}$, where
the exponential is evaluated element-wise.  Hence, equation \eqref{mak-alt} holds
and $c^\star(\mu)$ satisfies mass-action. Note that 
\eqref{convex-fix} is strictly convex, so for any feasible $\mu:=(r,r_0)$ there is 
a unique minimizer. That is, the mapping 
\begin{equation}
  \mu=(r,r_0) \rightarrow (v^\star(\mu),v^\star_0(\mu))
  \label{mapping}
\end{equation} 
is well defined.

If $\hat \mu = (r, r_0)$ is a fixed point of \eqref{mapping}, then
\eqref{opt-primal-feas} implies
\[
	 YDv^\star(\hat \mu)+YA^T\eta^+v^\star_0(\hat \mu) = YA^Tv^\star(\hat \mu) +
	 	YD\eta^-v_0^\star(\hat \mu)
\]
or equivalently
\[
    Y K v^\star(\hat \mu) = Y(A^T-D)v^\star(\hat \mu) = v^\star_0(YA^T\eta^+ - 
			YD\eta^-) = v^\star_0(\hat \mu)b.
\]
Therefore, if such a fixed point exists, the solution $v^\star(\hat\mu)$ at this
fixed point will satisfy
\begin{equation}
 YKv^\star=v^\star_0b.
  \label{scaled-fb}
\end{equation}
This implies that for $b=0$, both \eqref{fb} and \eqref{mak-alt} are
satisfied, and $c^\star$ defines a positive steady state of the system. 
For the case where $b\neq 0$, we will show how to construct a 
corresponding solution so that $\eqref{fb}$ holds.

For simplicity, we henceforth use $(v^\star,v^\star_0)$ and 
$y^\star$ to denote the optimal primal and dual solution, but acknowledge their
dependence on $\mu$.

\begin{theorem} 
	\label{fp-exist-map} 
	For any mass-conserving, mass-action chemical reaction network and any choice
	of rate constants $k>0$, there exist nontrivial fixed points for the mapping
	\eqref{mapping}.
\end{theorem} 

\begin{proof} 
	Brouwer's fixed-point theorem states that any continuous mapping from a
	convex and compact subset of a Euclidean space $\Omega$ to itself must have
	at least one fixed point. 

	Let $(v^\star,v^\star_0)$ be defined as in \eqref{opt-cond} and let $\gamma$
	be a positive fixed scalar. Define the set 
	\[
	\Omega = \left\{ (v,v_0)\in	\Re^{n+1} \; : \; (v,v_0)\geq 0, 
		\quad e^TYDv + e^TYA^T\eta^+v_0 = \gamma\right\},
	\]  
	where $e$ is defined in \eqref{consis}.  By Brouwer's fixed-point
	theorem, if the parameter $(r,r_0) \in \Omega$ ensures that the corresponding 
    optimal solution $(v^\star,v_0^\star) \in \Omega$, then there is a fixed point 
    such that the parameter and solution are equal, i.e., there exists a $\mu$ 
    such that $\mu = (r,r_0) = (v^\star(\mu),v_0^\star(\mu))$.  

	The set $\Omega$ is bounded and formed by an intersection of closed convex
	sets, and hence is convex and compact.  Moreover, the mapping $\mu\rightarrow
	(v^\star,v^\star_0)$ is continuous. Since problem \eqref{convex-fix} is
	feasible for any $\mu\in\Omega$, the mapping $\Omega \ni \mu \rightarrow
	(v^\star,v^\star_0)$ is well defined.

	To show that the image of $\Omega$ under the mapping $(r,r_0) \rightarrow
	(v^\star,v_0^\star)$	is in $\Omega$, first observe that by the bounds in
    \eqref{convex-fix}, $(v^\star,v^\star_0) \ge 0$.  Using \eqref{massbalanceb}, 
    \eqref{opt-primal-feas} and Definition \ref{consis}
	we have
	\begin{align}
	e^T YD v^\star + e^T YA^T \eta^+ v^\star_0 &= e^T YA^T r + e^T YD \eta^- r_0 \notag 
	\\ & = e^T YD r + e^T YA^T \eta^+r_0=\gamma, \notag 
	\end{align} 
	and thus $(v^\star,v_0^\star)\in \Omega.$
	Therefore, under the mapping $(r,r_0) \rightarrow (v^\star,v_0^\star)$,
	$(r,r_0) \in \Omega$ implies $(v^\star,v^\star_0) \in \Omega$, and the
	mapping must have a fixed point.  Moreover, since $\Omega$ does not contain
	the zero vector, the fixed point(s) are nontrivial.
    \qed
\end{proof}

Note that the value of $YDv^\star+YA^T\eta^+v^\star_0$ is the rate of
consumption of each chemical species and $YA^Tv^\star+YD\eta^-v^\star_0$ is the
rate of production of each chemical species. At the fixed point, the equality
$YDv^\star + YA^T\eta^+v^\star_0= YA^Tv^\star+YD\eta^-v^\star_0$ defines a
steady state. The set $\Omega$ defines the parameter $\gamma=e^T(YDv^\star +
YA^T\eta^+v^\star_0)$, and since the vector $e$ can be interpreted as an assignment
of relative mass to the species, $\gamma$ can be interpreted as the total
amount of mass that reacts per unit time at the steady state.  Therefore,
looking for fixed points in $\Omega$ corresponds to looking for steady states
where the amount of mass that reacts in the system is prescribed.

We have established the existence of a nontrivial fixed point $\mu$ of the
mapping $\Omega \ni \mu \rightarrow (v^\star,v^\star_0)\in \Omega$. Moreover,
we have shown that when the associated minimizer $(v^\star,v^\star_0)$ is
positive, it is a solution to \eqref{mak} and to
$YKv^\star=v^\star_0b$.  However, in the case when some entries of $v^\star$
are zero, the objective function of \eqref{convex-fix} is non-differentiable
and we cannot use the optimality conditions to show that \eqref{mak} holds. 


\subsection{Positive fixed points in weakly reversible single linkage networks}  
\label{section:single-linkage}

We now consider the case when the network is weakly reversible and formed by a single
linkage class and show that if $\hat \mu$ is a fixed point of the
mapping \eqref{mapping}, the minimizer $(v^\star(\hat
\mu),v^\star_0(\hat\mu))$, and therefore $\hat\mu$, is positive.

Lemma $\ref{maximum-support}$ below shows that if problem $\eqref{convex-fix}$ has a
feasible point with support $J$, the minimizer $(v^\star,v^\star_0)$ will have
support at least $J$. Lemma \ref{positive-feasible} uses the single
linkage class hypothesis to show that at a fixed point, there is a
positive feasible point.  These two Lemmas imply that at a fixed point
$\hat{\mu}$, the minimizer will be positive.  Finally, Theorem~\ref{thm:scaling}
shows that if $\hat{v}_0 \neq 1$ at the solution, we can construct another
solution for which $\hat{v}_0=1$. This establishes that there is a nontrivial
steady state for the network. 

To complete the argument we must prove Lemmas \ref{maximum-support},
\ref{positive-feasible} and Theorem~\ref{thm:scaling}.

\begin{lemma} 
	The support of any feasible point of problem \eqref{convex-fix} is a subset
	of the support of the minimizer $(v^\star,v^\star_0)$. 
	\label{maximum-support}
\end{lemma}

\begin{proof}
	Let $\tilde{v}\in \Re^{n+1}$ be any of the feasible points with the largest
	support and let $z$ be any feasible direction at $\tilde{v}$. By
	construction, for all $\alpha$ in some interval $[\ell,u]$ the points
	$v_\alpha:= \tilde{v} + \alpha z$ are non-negative and feasible.  The
	interval can be chosen so that when $\alpha=l$ and when $\alpha=u$, one new
	bound constraint becomes active.  This implies that $\supp(v_\ell)$ and
	$\supp(v_u)$ are strictly contained in $\supp(\tilde{v})$, and
	$\supp(v_\alpha) = \supp(\tilde{v})$ for $\alpha \in (\ell,u)$.
	
	Without loss of generality, we assume $\ell<0<u$, since $\ell$ and $u$ will
	not be of the same sign; if $\ell = 0$ and $u>0$, any point $v_{\alpha}$ can
	be written as a convex combination of $\tilde{v}$ and $\tilde{v}+uz$, and
	thus has support as large as $\tilde{v}$.

	Define the univariate function 
	\begin{equation}   
		g(\alpha) := \phi(\tilde{v} + \alpha z), 
		\label{univariate}
	\end{equation} 
	where $\phi$ is the objective function of \eqref{convex-fix}.  We will
	establish that as $\alpha\rightarrow l$ the derivative $g'(\alpha)\rightarrow
	-\infty$, and as $\alpha\rightarrow u$ the derivative $g'(\alpha)\rightarrow
	\infty$.  Thus, by the mean value theorem, there must exist a zero of the 
	function $g$ in the	interior of the interval $[l,u]$.  Since this function is
	strictly convex, if a stationary point exists in the interior of the
	interval, the function value at the stationary point must be smaller than at
	the boundary.

	Observe that if we let $d_i$, for $i\in[1,\dots n]$, be the diagonal entries
	of $D$ and $d_{n+1}=1$, we can write 
  \begin{align*}
	 g(\alpha) &=\sum_{i=1}^{n+1} (\tilde{v} + \alpha z_i) d_i\log(\tilde{v}_i + \alpha z_i).
  \end{align*} 

	An important observation is that if some entry $\tilde{v}_j=0$ then $z_j=0$,
	otherwise $v_\alpha$ would have a larger support for some $\alpha \neq 0$.
	This implies that $(v_\alpha)_j=0$ for all entries where $\tilde{v}_j=0$.  If
	we let $J$ be the set of nonzero entries of $\tilde{v}$, and $L$ be the
	subset of $J$ formed by the entries that tend to zero as $\alpha \rightarrow
	l$, then 
   \begin{align*} 
	 	g'(\alpha) &=\sum_{i\in J} z_i d_i(\log(\tilde{v}_i + \alpha z_i)) \\            
		&= \sum_{i\in (L^c\cap J)} z_id_i(\log{(\tilde{v_i}+\alpha z_i)}) + \sum_{i\in L}
		 z_id_i(\log{(\tilde{v_i}+\alpha z_i)}). 
	\end{align*} 
	As $\alpha \rightarrow l$, the first summation will approach a finite value.
	Since $z_i>0$ for all $i\in L$, the entries in the logarithm of the second
	sum tend to zero and the term will diverge to $-\infty$.

	Similarly, let $U$ be the subset of $J$ formed by the entries that tend to zero as 
	$\alpha\rightarrow u$. Observe that for these entries, $z_i<0$ and
  \[
    g'(\alpha) = \sum_{i\in (U^c\cap J)} z_i d_i(\log{(\tilde{v_i}+\alpha z_i)})
               + \sum_{i\in U}   z_i d_i(\log{(\tilde{v_i}+\alpha z_i)}).
  \]
	The first sum will tend to a finite value and the second will diverge to
	$\infty$. 
	
	Now, assume that for some $\mu$ there is a feasible point
	$(\tilde{v},\tilde{v_0})$ with larger support than the minimizer
	$(v^\star,v^\star_0)$ of problem \eqref{convex-fix}.  Since
	$(v^\star,v_0^\star)$ has smaller support, we can write  $(v^\star,
	v_0^\star) = (\tilde{v},\tilde{v}_0) + \alpha^\star z$ where $\alpha^\star$
	is on the boundary of the corresponding feasible interval.  By the
	previous argument, there is a value of $\hat \alpha \neq \alpha^\star$ in the
	interior of the interval such that $(v^\star, v_0^\star) =
	(\tilde{v},\tilde{v}_0) + \hat \alpha z$ has a lower function value than
	$(v^\star,v^\star_0)$, contradicting its optimality.

	Therefore, by the mean value theorm, there must exist a stationary point of
	$g$ strictly in the interior of the interval $[l,u]$ at which the function
	value is smaller than at the boundary.  Moreover, the optimal point will have
	at least the support of any feasible point.  
   \qed
\end{proof}



\begin{lemma}
	If the network is weakly reversible and formed by a single linkage class, when problem
	\eqref{convex-fix} is parametrized by a fixed point $\hat\mu$,  there
	exists a positive feasible point $(\hat v,\hat v_0)$.
\label{positive-feasible}
\end{lemma}

\begin{proof}
	Let problem \eqref{convex-fix} be parametrized with a fixed point $\hat\mu$,
	and let $(\hat v, \hat v_0)$ be both the minimizer and the fixed point.  We
	prove by contradiction that no entry of the minimizer $(\hat{v},\hat{v}_0)$
	can be zero. Observe that by the definition of $\Omega$ the origin is not
	contained in the set, and therefore the fixed point cannot be identically
	zero. 

	First, assume that $\hat{v}_0>0$ and observe that $\rho := (D^{-1}A^T\hat v + \eta^-\hat v_0, 0)$ is 
	a feasible point. Since $\eta^-$ was chosen to be positive and 
	$D^{-1}A^T$ has no zero columns, the support of $\rho$ are the first $n$ entries 
	of the vector. A convex combination of $(\hat v,\hat v_0)$ and $\rho$ will 
	be feasible and have full support.

	Now, assume that $\hat v_0 = 0$ and some entry of $\hat v$ is nonzero, and
	observe that $(D^{-1}A^T\hat{v},0)$ is feasible. A convex combination of
	$(D^{-1}A^T\hat{v},0)$ and $(\hat{v},0)$ is feasible and its support
	contains the union of the supports of the two vectors.  That is, for $\beta \in
	[0,1]$ the point  $(\tilde{v}, \tilde{v}_0) = \beta (D^{-1}A^T\hat{v},0) +
	(1-\beta)(\hat{v},0)$ is feasible, and using the fact that the support of
	$D^{-1}A^T\hat v$ is the support of $A^T	\hat v$ along with Lemma
	\ref{maximum-support},
	\begin{equation} 
		\left(\supp(\hat{v},0) \cup \supp(A^T\hat{v},0)\right) \subset \supp(\tilde{v},0).
	  \label{monotonesets} 
	\end{equation} 
	
	This relation can be used inductively to show that there is a feasible point
	with support at least as large as the union of the supports of
	$((A^T)^p\hat{v}, 0)$ for all positive powers of $p$.

	The weak reversibility and single linkage class hypotheses imply that for any pair
	$(i,j)\in [1,\dots,n]\times [1,\dots,n]$, there exists a power $p$ large
	enough such that $(A^T)^p_{ij}>0$.  More importantly, if $\hat v_j > 0$, then
	there exists a $p$ such that $[(A^T)^p \hat v_j]_i > 0$ for all $i$.  
	Therefore, if $\hat{v}_0 = 0$ and $\hat{v} \neq 0$, there is a 
	feasible point $(\tilde{v},0)$ such that $\tilde{v} > 0$.  
	
	Finally, if $\hat v > 0$ then there is a scalar $ 0<\alpha$ mall enough such
	that 
	\[
	0< \hat v - D^{-1}A^T\eta^+ \alpha,
	\]
	and then the equality  
	\[
		YD(\hat v - D^{-1}A^T\eta^+\alpha)+YA^T\eta^+\alpha = YD\hat v = YA^T\hat v
	\]
	implies that the positive point $(\hat v - D^{-1}A^T\eta^+\alpha,\alpha)$ is
	feasible.

	Therefore, if a network is weakly reversible and formed by a single linkage class, the
	problem \eqref{convex-fix} has a positive feasible point $(\hat v, \hat
	v_0)$.  
   \qed
\end{proof}

\begin{theorem}\label{thm:scaling}
	For a mass-conserving weakly reversible network with one linkage class there exists a
	concentration $c>0$ such that $YK\psi(c) = b$ if and only if $b$ is in the
	range of $YK$. 
\end{theorem}

\begin{proof}
	We have shown that there exist positive vectors $c \in \R^m, v \in \R^n$ such
	that $Y^T \log(c) = \log (v)$ and $Y K v = v_0 b$.  In other words, we have
	proven that there is a positive vector $c$ and a positive scalar $\alpha$
	such that $Y K \psi(c) = \alpha b$.  If we can construct a new
	concentration vector $\tilde c > 0$ that satisfies $\psi(\tilde c) =
	\frac{1}{\alpha}\psi(c)$, then
	\[
		YK\psi(\tilde c) = \frac{1}{\alpha}YK\psi(c) = b
	\]
	and the steady state concentration $\tilde c$ satisfies \eqref{fb} and
	\eqref{mak}.  

	First, we argue that the vector of all ones, $\1 \in \R^n$, is in the range
	of $Y^T$ when the network consists of a single linkage class and
	is mass-conserving. The condition of mass conservation implies that $e^T Y K = 0$, 
	or equivalently, $Y^T e \in \mathcal{N} (K^T)$.  Since $K^T$ is the 
	Laplacian matrix of a strongly connected graph, $\mathcal{N}(K) = \{\beta
	\1 :\; \beta \in \R\}$; thus, for some value $\hat \beta$, $Y^T e = \hat\beta
	\1$. 
	
	Now, observe that $\log\left(\frac{1}{\alpha}\psi(c) \right) =
	\log(\psi(c))-\1\log(\alpha) = Y^T\log(c) - \1\log(\alpha)$, where $\log(c)$
	is an entry-wise logarithm of the vector and $\log(\alpha)$ the scalar
	logarithm. Moreover, since $\1$ is in the range of $Y^T$, say $Y^T \delta =
	\1$ for some $\delta \in \R^m$, then $\log(\alpha)\1 =
	Y^T(\log(\alpha)\delta)$. Thus, if we define $\tilde c$ as the vector that
	satisfies  $\log(\tilde c) = \log(c) - \log(\alpha)\delta$ , then 
	\[
	Y^T\log(\tilde c) = Y^T\log(c) - \log(\alpha)\1,
	\] 
	which implies that 
	\[
	\psi(\tilde c) = \frac{1}{\alpha}\psi(c).
	\] 
	Therefore, the inhomogeneous system has a solution if the graph of
	the network is formed by a single linkage class, is weakly reversible
    and the network is mass-conserving, regardless of the kinetic parameters.
   \qed
\end{proof}


\section{Numerical experiments}
\label{sec:experiments}

We proceed to define the algorithm for finding positive steady states of weakly
reversible networks with or without material exchange across their 
boundary.

\subsection{Numerical method for finding fixed points}

Given an initial positive point $(\hat v, \hat v_0) \in \R^{n+1}$ and a small
tolerance $\tau$, we use the following fixed point iteration,
Algorithm~\ref{fp-iteration}, to find a parameter $\mu = (r, r_0)$ to
the problem \eqref{convex-fix} such that $(v^\star,v_0^\star) =
(r,r_0)$.

\begin{algorithm}
\caption{Fixed point iteration to find a steady-state concentration}
\label{fp-iteration}
\begin{algorithmic}[1]
  \STATE $(r,r_0) \gets (\hat v,\hat v_0)$
  \WHILE{$\|YK v^\star - (Y K \eta^+ + YD \eta^-) v_0^\star \|_\infty>\tau$}
  \STATE $(v^\star,v^\star_0)\gets \text{unique solution of \eqref{convex-fix}} $
		\label{min-step}
	\STATE $(r,r_0) \gets \frac{1}{2}(r,r_0) +\frac{1}{2}(v^\star,v^\star_0)$
  \ENDWHILE
  \label{fixpoint-alg}.
\end{algorithmic}
\end{algorithm}

Step \ref{min-step} of Algorithm~\ref{fp-iteration} requires
solving the linearly constrained convex optimization problem
\eqref{convex-fix}. Our implementation uses PDCO \cite{pdco} to
solve this problem.

Provided that at each iteration $k$, the unique solution of \eqref{convex-fix}
satisfies $v^\star(\mu_k)>0$ and the minimization is solved with sufficient
accuracy, the optimality conditions for \eqref{convex-fix} will imply that for
all $k = 1, 2, 3, \dots$,
\[
	\|Y^Ty^\star-\log(v^\star(\mu_k))\|_\infty \leq \epsilon,
\] 
for some small value of $\epsilon$, where $y^\star$ is the Lagrange
multiplier of the linear equality constraint at the solution that corresponds
to the logarithm of the concentrations. Thus, if the iteration
converges to a fixed point $(r,r_0) = (v^\star,v_0^\star)$, then
\eqref{mak} will be satisfied to precision $\epsilon$ and \eqref{fb}
to precision $\tau$.

Algorithm~\ref{fp-iteration} has been tested extensively on randomly generated
networks with noteworthy success. The results of our experiments are shown in
section~\ref{scn:convergence}.

\subsection{An example network}
\label{example-network}

\begin{figure}%[!htbp]    %%% 1
   %\sidecaption
   \centering
   \includegraphics[width=3in]{paperNetwork}
   \caption{Example weakly reversible network with two linkage classes}
   \label{fig:network-small}
\end{figure}

In this section we consider the weakly reversible network shown in
Figure~\ref{fig:network-small}.  The number of complexes is $n = 7$, the number of
linkage classes is $\ell = 2$.
%, and the stoichiometric subspace $S =
%\operatorname{span}\{ A+E-C, C-A-D, B-C\}$ has dimension $s = 3$.  Therefore,
%the deficiency of this network is $\delta = 7 - 2 - 3 = 2$, and hence neither
%of the Deficiency 0-1 theorems \cite{deficiency0,deficiency1} can be applied
%to calculate equilibrium points. 
Since this network is weakly reversible, intuition suggests that a
non-zero steady state exists.  We use Algorithm~\ref{fp-iteration} to solve for
the fixed point described in section~\ref{section:fp-model}, obtaining a
positive steady state.  Figure~\ref{fig:ConcentrationVsIteration} illustrates
the convergence of the fixed point iterations to the steady state.

\begin{figure}%[!htbp]    %%% 2
   \sidecaption
   \includegraphics[width=4.5in]{ConcentrationVsIterationExample}
   \caption{Concentration convergence with iteration}
   \label{fig:ConcentrationVsIteration}
\end{figure}

Figure~\ref{EquilibriumVsTotalMass} illustrates the change in steady state as a
function of the \emph{total mass} in the system, where total mass is defined as
$\gamma = e^T(YD v^\star + YA^T \eta^+ v^\star_0)$, as described in
Theorem~\ref{fp-exist-map}.  The experiment shows that as the total mass increases,
species $A$, $B$ and $C$ adjust linearly to the additional mass, while species
$D$ and $E$ stay at the same levels.  This linear growth in species $A, B$ and
$C$ can be explained analytically by the fact that the vector $\mathbf{1}$ lies
in the range of $Y^{T}$. 

\begin{figure}%[!htbp]   %%% 3
   \sidecaption
   \includegraphics[width=4.5in]{EquilibriumVsTotalMassExample}
   \caption{Equilibrium dependence on total mass} 
   \label{EquilibriumVsTotalMass}
\end{figure}

\subsection{Generating suitable networks}

This section describes the sampling scheme used to generate random mass-conserving
weakly reversible chemical reaction networks with a prescribed number of
linkage classes.  The output of the method is a network with $n$
complexes, $m$ species, and $\ell$ strongly connected components, where $\ell$
is the desired number of linkage classes.

First, we iteratively generate Erd\H{o}s-R\'{e}yni graphs% 
\footnote{An Erd\H{o}s-R\'{e}yni graph is a directed unweighted graph. Each
edge is included with probability $p$ and all edges are sampled \emph{iid}.} % 
with $m$ nodes until we sample a graph with $\ell$ strongly connected
components; call this graph $\hat G( \hat V, \hat E)$.  We give each edge in 
$\hat E$ a weight of an independent and uniformly distributed value in the
range $(0,10]$.  These edge weights represent the reaction rates between
complexes.

To generate the stoichiometry, we define a parameter $q_{\max}$ as the maximum 
number of species in each complex. Each complex is constructed with a random
sample of $q$ species, where $q$ is a random integer in $[1,q_{\max}]$.  All samples
are drawn uniformly and independently.  Finally, we assign the multiplicity of
each species in a complex with independent samples of the absolute value of a
standard normal unit variance distribution. To ensure mass is conserved, we
normalize the sum of the stoichiometry of the species that participate in a
complex to one, so that $Y^T\1 = \1$ and $K^TY^T \1 = \0$. 

\subsection{Convergence of the fixed point algorithm}
\label{scn:convergence} 

This section illustrates the convergence of Algorithm~\ref{fp-iteration} on
large weakly reversible networks that consist of either a single linkage class 
or multiple linkage classes, for cases where there is no material exchange
across the boundary ($b=0$) and cases where there is material exchange
across the boundary ($b=YK\eta$ for some $\eta \neq 0$).

Algorithm~\ref{fp-iteration} produces sequences that, up to a small tolerance,
satisfy \eqref{mak} at every iteration. Ideally, the infeasibility with 
respect to \eqref{fb} also monotonically decreases until convergence.  Our
extensive numerical experiments indicate that this is in fact the behavior for
closed networks that are weakly reversible.

\begin{figure}%[!hbtp]   %%% 4
   %\sidecaption
   \centering 
   \includegraphics[width=0.8\textwidth]{InfeasibilityVsIteration}
   \caption{Typical infeasibility of \eqref{fb} vs.\ Iteration for network with a
            single linkage class ($b=0$)} 
   \label{fig:typical-infeas-single} 
\end{figure}

\begin{figure}%[!htbp]   %%% 5
   \centering
   \includegraphics[width=0.78\textwidth]{InfeasibilityVsIterationMultiple}
   \caption{Typical infeasibility of \eqref{fb} for network with
            two linkage classes ($b=0$)}
   \label{fig:typical-infeas-multiple}
\end{figure}

Figure~\ref{fig:typical-infeas-single} displays the sequence of the
infeasibilities $\|YKv_k\|_\infty$ at each iteration $k$ in
Algorithm~\ref{fp-iteration}, for a network with a single
linkage class, $50$ species and $500$ complexes, where at
most $10$ species participate in each
complex. Figure~\ref{fig:typical-infeas-multiple} displays the
analogous sequence for a network of equal size and two
linkage classes.  We have observed this (apparently linear)
convergence rate consistently over all generated networks, regardless
of the number of linkage classes.

\begin{figure}%[!hbtp]   %%% 4
   %\sidecaption
   \centering 
   \includegraphics[width=0.8\textwidth]{InfeasibilityVsIterationOpen}
   \caption{Typical infeasibility of \eqref{fb} vs.\ Iteration for network with a
            single linkage class with material exchange ($b\neq0$)} 
   \label{fig:typical-infeas-single-open} 
\end{figure}

\begin{figure}%[!htbp]   %%% 5
   \centering
   \includegraphics[width=0.78\textwidth]{InfeasibilityVsIterationOpenMultiple}
   \caption{Typical infeasibility of \eqref{fb} for network with
            two linkage classes and material exchange ($b\neq0$)}
   \label{fig:typical-infeas-multiple-open}
\end{figure}

Figure~\ref{fig:typical-infeas-single-open} displays
the sequence of the infeasibilities $\|YK\eta_k-b\eta_0\|_\infty$ 
at each iteration $k$, for the single linkage network with material 
exchange across its boundary. Figure~\ref{fig:typical-infeas-multiple-open} 
displays the analogous sequence for the multiple linkage network with material
exchange across its boundary. Notably, the linear convergence rate
is preserved.

\begin{figure}%[!htbp]   %%% 6
   \centering
   \includegraphics[width=0.78\textwidth]{SingleNetAvgIterationsVsNetSize} 
   \caption{Average number of iterations for single linkage class
            networks ($b=0$)}
   \label{fig:iteration-count-simple} 
\end{figure}

\begin{figure}%[!htbp]   %%% 7
   \centering
   \includegraphics[width=0.78\textwidth]{MultipleNetAvgIterationsVsNetSize} 
   \caption{Average number of iterations for networks with two linkage
            classes ($b=0$)}
   \label{fig:iteration-count-multiple} 
\end{figure}

%\clearpage

We have also investigated the number of iterations necessary for
Algorithm~\ref{fp-iteration} to converge on closed networks of different
sizes, with either one or two linkage classes.
Figures~\ref{fig:iteration-count-simple} and
\ref{fig:iteration-count-multiple} display the mean number of
iterations necessary for convergence on networks ranging from $100$ to
$5000$ complexes, where each average is taken over $20$ instances per
network size. Notably, the average number of iterations increases less
than $10\%$ as the network size grows fifty-fold.

In future work, we plan to prove theoretical results on the existence of
positive equilibria for chemical reaction networks with multiple
linkage classes.  However, our comprehensive numerical experiments
seem to indicate that even for networks with more than one linkage
class, there exists at least one positive fixed point of problem
\eqref{convex-fix}, and the iterates of Algorithm~\ref{fp-iteration} converge
to such a fixed point.

%\clearpage

\enlargethispage{3\baselineskip}

\begin{acknowledgements}
We gratefully acknowledge Anne Shiu and her student for helpful
discussions and for directing us to \cite{Deng}.
This research was supported in part by the U.S. Department of Energy (Office
of Advanced Scientific Computing Research and Office of Biological
and Environmental Research) as part of the Scientific Discovery
Through Advanced Computing program, grant DE-SC0002009,
and by NSF grant GOALI 0800151.
\end{acknowledgements}

\frenchspacing
%\nocite{*} 
\bibliographystyle{spmpsci} 
\bibliography{fixpoint}{}


\end{document}
